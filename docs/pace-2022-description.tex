
\documentclass[a4paper,UKenglish,cleveref, autoref, thm-restate]{lipics-v2021}
%This is a template for producing LIPIcs articles. 
%See lipics-v2021-authors-guidelines.pdf for further information.
%for A4 paper format use option "a4paper", for US-letter use option "letterpaper"
%for british hyphenation rules use option "UKenglish", for american hyphenation rules use option "USenglish"
%for section-numbered lemmas etc., use "numberwithinsect"
%for enabling cleveref support, use "cleveref"
%for enabling autoref support, use "autoref"
%for anonymousing the authors (e.g. for double-blind review), add "anonymous"
%for enabling thm-restate support, use "thm-restate"
%for enabling a two-column layout for the author/affilation part (only applicable for > 6 authors), use "authorcolumns"
%for producing a PDF according the PDF/A standard, add "pdfa"

%\graphicspath{{./graphics/}}%helpful if your graphic files are in another directory

\bibliographystyle{plainurl}% the mandatory bibstyle

\title{PACE 2022 Solver Description: An Exact Solver for Directed Feedback Vertex Set} %TODO Please add

% \titlerunning{PACE 2022 Solver Description} %TODO optional, please use if title is longer than one line


\newcommand{\universityOfUtah}{School of Computing, University of Utah, USA}
\author{Yosuke Mizutani}{\universityOfUtah}{yos@cs.utah.edu}{https://orcid.org/0000-0002-9847-4890}{}

\authorrunning{Y. Mizutani} %TODO mandatory. First: Use abbreviated first/middle names. Second (only in severe cases): Use first author plus 'et al.'

\Copyright{Yosuke Mizutani} %TODO mandatory, please use full first names. LIPIcs license is "CC-BY";  http://creativecommons.org/licenses/by/3.0/

\ccsdesc[100]{Theory of computation → Graph algorithms analysis} %TODO mandatory: Please choose ACM 2012 classifications from https://dl.acm.org/ccs/ccs_flat.cfm 

\keywords{Directed feedback vertex set, PACE 2022} %TODO mandatory; please add comma-separated list of keywords

\category{} %optional, e.g. invited paper

\relatedversion{} %optional, e.g. full version hosted on arXiv, HAL, or other respository/website
%\relatedversiondetails[linktext={opt. text shown instead of the URL}, cite=DBLP:books/mk/GrayR93]{Classification (e.g. Full Version, Extended Version, Previous Version}{URL to related version} %linktext and cite are optional

\supplement{
    Code submitted to the competition: \url{https://doi.org/10.5281/zenodo.6604875}.
    Code repository on GitHub: \url{https://github.com/mogproject/dfvs-2022}
}
%optional, e.g. related research data, source code, ... hosted on a repository like zenodo, figshare, GitHub, ...
%\supplementdetails[linktext={opt. text shown instead of the URL}, cite=DBLP:books/mk/GrayR93, subcategory={Description, Subcategory}, swhid={Software Heritage Identifier}]{General Classification (e.g. Software, Dataset, Model, ...)}{URL to related version} %linktext, cite, and subcategory are optional

%\funding{(Optional) general funding statement \dots}%optional, to capture a funding statement, which applies to all authors. Please enter author specific funding statements as fifth argument of the \author macro.

%\acknowledgements{I want to thank \dots}%optional

%\nolinenumbers %uncomment to disable line numbering

\hideLIPIcs  %uncomment to remove references to LIPIcs series (logo, DOI, ...), e.g. when preparing a pre-final version to be uploaded to arXiv or another public repository

%Editor-only macros:: begin (do not touch as author)%%%%%%%%%%%%%%%%%%%%%%%%%%%%%%%%%%
% \EventEditors{John Q. Open and Joan R. Access}
% \EventNoEds{2}
% \EventLongTitle{42nd Conference on Very Important Topics (CVIT 2016)}
% \EventShortTitle{CVIT 2016}
% \EventAcronym{CVIT}
% \EventYear{2016}
% \EventDate{December 24--27, 2016}
% \EventLocation{Little Whinging, United Kingdom}
% \EventLogo{}
% \SeriesVolume{42}
% \ArticleNo{23}
%%%%%%%%%%%%%%%%%%%%%%%%%%%%%%%%%%%%%%%%%%%%%%%%%%%%%%

\newtheorem{rulex}[theorem]{Rule}

\begin{document}

\maketitle

%TODO mandatory: add short abstract of the document
\begin{abstract}
This note describes our submission to the Exact track of PACE 2022:
 \textsc{Direct Feedback Vertex Set} (\textsc{DFVS})\footnote{%
\texttt{https://pacechallenge.org/2022/}}.
Our solver heavily depends on the \textsc{Vertex Cover} solver \textsf{WeGotYouCovered}
 \cite{hespe2020wegotyoucovered},
 as we observe that \textsc{DFVS} can be reduced to \textsc{Vertex Cover}
 if all edges in the given digraph are bidirected.
%
Further, our program comprises two solvers: the heuristic solver and the exact solver.
The heuristic solver \textit{guesses} a high-quality feasible solution via random search,
 and the exact one performs an exhaustive branch-and-bound search guided by the \textit{guess} solution.
\end{abstract}

\section{Background}
\label{sec:background}

In addition to standard graph-theoretic notation (e.g. found in \cite{diestel2017graph}),
 we define a few terms.
Given a directed graph, or a \textit{digraph}, $G=(V,E)$,
 we say an edge $uv \in E$ is \textit{strong} if bidirected, i.e. $\{uv, vu\} \in E$,
 and otherwise, $uv$ is \textit{weak}.
Thus, $E$ is partitioned into strong edges, denoted by $E_\text{strong}$,
 and weak edges, denoted by $E_\text{weak}$.
%
Any digraph can be seen as a pair of $(G_\text{strong}, G_\text{weak})$,
 where $G_\text{strong}$ is an undirected graph $(V,E_\text{strong})$
 and $G_\text{weak}$ is a digraph $(V,E_\text{weak})$.

Important graph operations in our algorithm include vertex deletion ($G-v$),
 edge contraction ($G/e$) and vertex ignoring (\textsf{ignore}($v$)),
 as defined in \cite{levy_contraction_1988} and \cite{hen-ming_lin_computing_2000}.
%
The \textsf{ignore} operation adds edges from all in-neighbors of $v$ to all out-neighbors of $v$,
 and then removes $v$.
Formally, for a digraph $G=(V,E)$ and a vertex $v \in V$, $\textsf{ignore}(v)$ results in
 $G' - v$, where $G'=(V, E \cup \{uw: u \in N^-(v), w \in N^+(v) \})$.

Given a digraph $G=(V,E)$,
 the \textsc{Direct Feedback Vertex Set} (\textsc{DFVS}) problem asks for a smallest vertex set $S \subseteq V$
 such that $G-S$ is acyclic, that is, the digraph obtained by removing $S$ from $G$ does not contain
 any directed cycles.
The \textsc{Vertex Cover} problem is defined as: given an (undirected) graph $G=(V,E)$,
 find a smallest vertex set $S \subseteq V$ such that for every edge $e \in E$, $e \cap S \neq \emptyset$.
We relate those problems as follows.

\begin{proposition}\label{thm:1}
    Given a digraph $G=(V,E)$, its strong and weak graphs $G_\text{strong}=(V,E_\text{strong})$,
    $G_\text{weak}=(V,E_\text{weak})$, respectively,
    when $E_\text{weak}=\emptyset$, any solution to \textnormal{\textsc{Vertex Cover}} with $G_\text{strong}$
    is also a solution to \textnormal{\textsc{DFVS}} with $G$.
\end{proposition}

We use \cref{thm:1} for two purposes: (1) finding the exact solution when $E_\text{weak}=\emptyset$ and
 (2) computing the lower-bound of the solution size otherwise.


\section{Brief Description of Algorithm}

Our solver iteratively applies reduction rules and performs two phases of computation.
The first phase, or guess, aims to find the best solution quickly;
 here, we use random search enhanced with local search.
The second phase performs a branch-and-bound exhaustive search utilizing the guess solution as guidance.

\subsection{Reduction Rules}

We employ all the reduction rules from \cite{levy_contraction_1988} and \cite{hen-ming_lin_computing_2000}.
It is known that a solution to \textsc{DFVS} with $G$ is the union of each strongly connected component of
 $G$ \cite{hen-ming_lin_computing_2000},
 so our reduction process takes a digraph $G$ and returns a set of digraphs $\{G_1,\ldots\}$,
 possibly adding new vertices to solution.

\begin{itemize}
    \item \textbf{IN0.} Given a vertex $v$ with in-degree zero, remove $v$.
    \item \textbf{OUT0.} Given a vertex $v$ with out-degree zero, remove $v$.
    \item \textbf{LOOP.} Given a vertex $v$ with a self-loop, add $v$ to solution and remove $v$.
\end{itemize}

The following rules assume $G$ has no self-loops.

\begin{itemize}
    \item \textbf{IN1.} Given a vertex $v$ with in-degree one, i.e. $N^-(v)=\{u\}$ for $v \neq u$,
        contract edge $uv$.
    \item \textbf{OUT1.} Given a vertex $v$ with out-degree one, i.e. $N^+(v)=\{u\}$ for $v \neq u$,
        contract edge $vu$.
    \item \textbf{PIE.} Given a weak edge $uv \in E_\text{weak}$ such that $u$ and $v$ are not strongly connected in $G_\text{weak}$,
        remove $uv$ from $G$.
    \item \textbf{CORE.} Let $C \subseteq V$ be a clique in $G_\text{strong}$.
        If there is a vertex $v \in C$ such that $N_G^-(v) \cup N_G^+(v) \subseteq C$, then
        remove $C \setminus \{v\}$.
    \item \textbf{DOME.} Given a weak edge $uv \in E_\text{weak}$ such that
        $N_{G_\text{weak}}^-(u) \subseteq N_G^-(v)$ or
        $N_{G_\text{weak}}^+(u) \subseteq N_G^+(v)$, remove $uv$.
\end{itemize}

Additionally, we introduce the following rule underpinned by \cref{thm:1}.

\begin{itemize}
    \item \textbf{VC.} If $E_\text{weak} = \emptyset$, compute solution to \textsc{Vertex Cover} with $G_\text{strong}$
    and add it to the solution to \textsc{DFVS}. Then, remove all vertices.
\end{itemize}

Note that all reduction rules except \textbf{VC} can be performed in polynomial time\footnote{With respect to the graph size.}.

%-------------------------------------------------------------------------------
%
%-------------------------------------------------------------------------------
\subsection{First Phase: Random Search}

In this phase, our solver iteratively picks a vertex or an edge and decides to include in solution or not
(details in the branching section).
The solver applies reduction rules every time the graph is modified.
Once a feasible solution to the original instance has be found, it runs a local search to see
 if there is a better solution.
Different branching strategies and pseudorandom number generator seeds are tried until the time limit exceeds.

\textbf{Branching.} The solver uses two branching strategies: (1) vertex branching and (2) edge branching.
Both strategies take two parameters $\sigma \in \{-1,1\}$ and $\mu \in \mathbb{R}$, $0 \leq \mu < 1$.
Ties are broken randomly.
%
The vertex branching picks a vertex $v$ such that $\sigma \cdot \deg_{G_\text{weak}}(v)$ is minimized.
Then, with probability $\mu$, (1a) include $v$ in the solution and remove $v$, and otherwise, (1b) ignore $v$.

The edge branching first picks a weak edge $uv \in E_\text{weak}$ such that
 $\sigma (\deg_{G_\text{weak}}(u) + \deg_{G_\text{weak}}(v))$ is minimized.
 Ties are broken randomly.
Then, with probability $\mu$, (2a) add edge $vu$, and otherwise, (2b) ignore $u$ and $v$.
%
The idea here is that if either $u$ or $v$ is in the solution, the instance is equivalent to the one where
$uv$ is strong.
Otherwise, we can safely ignore $u$ and $v$.
Notice that both (2a) and (2b) decreases the number of weak edges,
 so the graph eventually becomes strong and the reduction rule \textbf{VC} will be applied.

\textbf{Local search.} Once the solver finds a feasible solution,
 it tries the following changes to see if there is a better solution:
 (1) removing one vertex from solution and
 (2) removing two vertices $u$ and $v$ from solution and add another vertex $w$.
%
Here, the candidates for $w$ can be found by two BFS;
 after removing all vertices in the solution except $u$ and $v$,
 $w$ must be in both the smallest cycle including $u$ and the smallest cycle including $v$.

\vspace*{1em}

The best guess (the one with the least solution size) along with its branching history
 is preserved for the second phase.

% \enlargethispage{\baselineskip}

\subsection{Second Phase: Exhaustive Search}

In the second phase, our solver runs a branch-and-bound procedure guided by the best guess (if such one exists).
%
Let us call the guess solution $S$.
First, we seek the vertices $S' \subseteq S$ that are also in the exact solution as follows:
for each $v \in S$, we compute the lower bound of the graph obtained by \textsf{ignore}($v$).
If it is not less than $|S|$, we include $v$ in our exact solution.
We use the same reduction rules throughout the algorithm.
The lower-bound function also involves the \textsc{Vertex Cover} solver and
 ideas inspired by \cite{hen-ming_lin_computing_2000}.

\textbf{Branching.} In this phase, the solver considers all branches.
It chooses the strategy for the best guess, but tailored to the guess solution;
 if there is a vertex or an edge used during the best guess, we make the same decision in the same order.
Otherwise, it picks a branch target as performed in the first phase.

\textbf{Lower bounds.} At every search node, we compute a lower bound of the solution size as follows.
Let $V_\text{strong} \subseteq V$ be the set of vertices incident to any strong edge.
Then, a lower bound can be found by $\alpha + \beta$,
 where $\alpha$ is the solution size of \textsc{Vertex Cover} with $G_\text{strong}$,
 and $\beta$ is the (vertex-disjoint) cycle-packing number of $G - V_\text{strong}$.
Packing is done with sampling. Randomly pick a vertex and obtain the shortest cycle by BFS.
Remove the vertices forming that cycle and then repeat.
%
Reduction rules are repeatedly applied during the computation.

%%
%% Bibliography
%%

%% Please use bibtex, 

\bibliography{pace-2022-description}

\end{document}
